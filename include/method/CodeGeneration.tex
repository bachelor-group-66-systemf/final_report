\section{Code Generation to LLVM IR (INCREDIBLY ROUGH DRAFT :))}
\todo[inline]{MYCKET JAG FORM}
One of the last steps in the compilation chain is creating the code that will  
be run on the target machine.

\subsection{The choice of approach}
When targeting LLVM you have two major choices for generating your IR code; 
you can either generate the IR code on your own, or you can use the LLVM library.

For this project, it was decided to generate the IR code ourselves, as 
the LLVM libraries that existed for Haskell, LLVM-HS specifically, did not compile 
with the newer versions of Haskell, nor does it work with the newer versions of 
LLVM as it targets LLVM 9 (released 2019), and for this project it was decided to use a later
version, specifically LLVM 15 (released 2022), as it would be easier to work against
as it is widely available compared to the old releases. At the time of writing this report,
there is a branch of the LLVM-HS library targeting LLVM 15, but due to something something being new and untested :)

\subsection{Mapping Churf code to LLVM IR}
Translating the AST into LLVM IR code works quite well due to the immutable nature of lambda calculus,
and because most functions are simply expression chains that return their final values.

\todo[inline]{Mapping Functions to LLVM}
\begin{verbatim}
func : Int -> Int -> Int
func x y = x + y
\end{verbatim}
\begin{verbatim}
define i64 @func(i64 %x, i64 %y) {
    	%1 = add i64 %x, %y
    	ret i64 %1
}
\end{verbatim}

\todo[inline]{Mapping Expressions to LLVM}
\begin{verbatim}
f = (funB 1 2 3) + (funA 3)
funA x = x + 5
funB x y z = x + y + z
\end{verbatim}
\begin{verbatim}
define i64 @f() {
    	%1 = call i64 @funB(i64 1, i64 2, i64 3)
    	%2 = call i64 @funA(i64 3)
    	%3 = add i64 %1, %2
    	ret i64 %3
}

define i64 @funA(i64 %x) {
    	%1 = add i64 %x, 5
    	ret i64 %1
}

define i64 @funB(i64 %x, i64 %y, i64 %z) {
    	%1 = add i64 %x, %y
    	%2 = add i64 %1, %z
    	ret i64 %2
}
\end{verbatim}

\todo[inline]{Using unnamed variables in LLVM}
To create temporary variables one can use LLVM's unnamed temporary variables.
These are variables that are named $\%n$

\todo[inline]{The inability to output constants}
Do not allocate unnecessary variables, and instead use values directly if possible.

\todo[inline]{Translating EApp}
\begin{verbatim}
appEmitter :: ExpT -> ExpT -> [ExpT] -> CompilerState ()
appEmitter e1 e2 stack = do
    let newStack = e2 : stack
    case e1 of
        (MIR.EApp e1' e2', _) -> appEmitter e1' e2' newStack
        (MIR.EVar name, t) -> do
            args <- traverse exprToValue newStack
            vs <- getNewVar
            funcs <- gets functions
            consts <- gets constructors
            let visibility =
                    fromMaybe Local $
                        Global <$ Map.lookup name consts
                        <|>
                        Global <$ Map.lookup (name, t) funcs
                args' = map (first valueGetType . dupe) args
                call = Call FastCC (type2LlvmType rt) visibility name args'
            emit $ SetVariable vs call
        x -> error $ "The unspeakable happened: " <> show x
\end{verbatim}
\todo[inline]{Mapping Types to LLVM}
Primitive types such as Ints and Chars are replaced with their LLVM counterparts (i64/i8). 
\todo[inline]{Data Types}
Data types are translated to tagged unions, ala the Rust way
\todo[inline]{Representing Algebraic Data Types}
\todo[inline]{Constructing Constructors}
\todo[inline]{Case Expressions} 
\subsection{Integration of Garbage Collector and Compilation}
\todo[inline]{Integrating Garbage collection}
\todo[inline]{Chucking the product to Clang}