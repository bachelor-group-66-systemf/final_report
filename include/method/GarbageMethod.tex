
% -------------------------------------------
% Jag har bara stolpat upp lite punkter som jag tror kan va bra att nämna
% men dem får gärna ändras
% -------------------------------------------

\section{Creating a garbage collector}

Managing system resources is not necessarily a job for the compiler, but for the
program it outputs. In most modern languages, like C\# or Java, garbage collection
or other resource management types are part of the languages' standard runtime
libraries. This implies different problems and design choices than those of the
compiler. Because of this, a choice was made early in the process to write the
garbage collection library in a different host language. Haskell is what is considered
to be a modern high-level language that abstracts several areas, like manual memory
management, away from the
developer for convenience. This makes it significantly harder to precisely deal
with system resources closer to the operating system or hardware. Because of this,
C++ was chosen as the host language.

\subsection{Designing the runtime library}
Inspiration was taken from the GNU C standard library, which exposes functions
like malloc and free, to allocate and free heap memory respectively, but hides
implementation details from the developer. That is why the garbage collection
library was designed similarly. Since the point of the library was to manage
memory automatically, the only functionality needed was a way for the developer
to allocate memory.\\

Due to the lack of experience in memory management among the group members,
a decision was made to emulate the heap to minimize the effects on the whole
heap segment of the program, and instead encapsulate the memory operations
into the emulation.\\

% JAG VILL SKRIVA KOD INTE TEXT :(

Modern implementations of garbage
collectors can take up to 38\% of a program's execution time [source here I found earlier]. Because of this,
the garbage collector was designed to only collect memory when necessary. That is
when there is not enough free heap memory to directly handle the allocation request.
This reduced the average time spent by the allocation function and improved performance
but came with the side effect of unused memory spending more time in memory than
required. It was deemed to not be a problem since the library emulates the heap
and thus the total memory used by the program as seen by the operating system is
constant.

\subsection{Algorithm implementation}


\subsection{Integration with the compiler}
Because the garbage collector is a runtime library, it was decided to be linked with
the compiler output after compilation. This was a straightforward decision after
the capabilities of LLVM IR was recognized. LLVM IR allows the developer to call
functions from libraries loaded in memory, without  

The library source code was compiled to
a static library which was later linked with the compiled executable by the
operating system's linker (a program defined by the operating system to link
dynamic and static libraries for a program or library).
