very rough :))

A part of creating a programming language is making a representation of runnable by computers,
for this task, some sort of assembly language is a suitable solution.
\\

An assembly language is a low-level programming language that closely represents a machine's actual 
CPU instructions. These languages are often quite simple, consisting of CPU instructions known as opcodes,
labels to allow for branching code, and in some cases, features like macros.
\todo[inline]{insert example assembly code?}
\\

There are a variety of available assembly languages, including the x86 family, the ARM assembly languages, and more.
However, these flavors of assembly come with a big flaw, which is being platform dependant, and in some cases 
CPU model dependant. For example; a program written in ARMv7 assembly will not run on an x86 CPU, and the same goes
for x86 assembly, which can not run on an ARMv7 CPU.
\\

The creation of LLVM started in 2000, and while it was initially developed to explore dynamic compilation techniques, 
it has over time evolved to be quite a large compiler framework encompassing many different smaller projects.
One of these projects is the high-level intermediate representation assembly language LLVM IR, which is a 
high-level assembly language that can be compiled into many different assembly languages, 
such as the x86 family or the ARM languages mentioned before.

This intermediate language closely resembles the lower-level assembly languages, while also shaking things up
by having a type system(a rarity among assembly languages), type declarations, function definitions, and more.
\\

LLVM IR is written in single static assignment form (SSA), which just adds the rule that variables may only 
be assigned once. This guarantee makes optimization of the IR code easier, as unused and unnecessary variables
can easily be optimized away and quite a few other optimization algorithms make use of it,
although it comes at the cost of being harder to translate to compared to other assembly languages, depending on
the source language.
\todo[inline]{insert sources}
\todo[inline]{insert example}