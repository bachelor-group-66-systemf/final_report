The Curry-Howard correspondence relates the simply typed $\lambda$-calculus 
to intuitionistic logic \citep{smolka-notes}. 
The correspondence describes the construction and deconstruction of types, for instance,
the function connective $\rightarrow$ constructs a function \citep{howard-1980}.
In logic, this corresponds to introduction and elimination rules. Also, the function connective $\rightarrow$ corresponds to implication $\implies$. 
\\
\\
Type systems can be designed by combining introduction and elimination rules.
For example, abstraction $\lambda x.e$ and application $e_1 e_2$ can be type checked by the introduction and elimination rules for $\rightarrow$. 
\begin{center}
\vskip 1em
\AxiomC{$\Gamma, (x : A) \vdash e : B$}
\RightLabel{$\rightarrow$I}
\UnaryInfC{$\lambda x.e : A \rightarrow B$}
\DisplayProof
\hskip 1.5em
\AxiomC{$\Gamma \vdash e_1 : A \rightarrow B$}
\AxiomC{$\Gamma \vdash e_2 : A$}
\RightLabel{$\rightarrow$E}
\BinaryInfC{$e_1 e_2 : B$}
\DisplayProof
\vskip 1em
\end{center}
The judgments use the form $\Gamma \vdash e : A$, read "in context $\Gamma$, term e has type A". 
$\Gamma$ gives the type of the free variables in $e$, and
the notation $\Gamma , (x : A)$ is the context that extends $\Gamma$ by associating the identifier $x$ with type $A$.
Notice that the introduction rule introduces the connective
in the conclusion, while the elimination rule eliminates it from the premises. 
The type-checking algorithm can be mechanically derived from the rules. To check if the abstraction $\lambda x.e$ has type $A \rightarrow B$: extend the context with $x : A$ and check if the proposition $e : B$ holds.
Similarly, the application $e_1 e_2$ have type $B$ if $e_1$ and $e_2$ have type $A \rightarrow B$ and $A$ respectively.

\todo[inline]{Type systems are tradeoffs}
\todo[inline]{Type checking is the process of checking if a given proof holds}
\todo[inline]{Type inference is the process of creating a proof for a given proposition}





\subsection{Hindley-Milner}
\begin{itemize}
    \item The Hindley-Milner type system presents a fairly simple way system for type inference and type checking in polymorphic lambda calculus.
    \item One advantage is complete type inference, no annotations are required at all
    \item It has show faily simple to extend to user defined data types as well as case expressions.
    \item Error messages are hard to report well, as an ill-typed program is first detected in the unification part
\end{itemize}

\emph{This text assumes we have already argued for why a type system is favourable}


\subsection{Bidirectional}
\begin{itemize}
    \item popular for its scalability, error reporting, and ease of implementation \citep{bidir-gadts}.
    \item Bidirectional type checking uses two modes: type checking and type synthesis. Checking is easier and allows for a more expressive type system but it requires explicit annotations. Synthesizing a program is harder and undecidable for some language features \citep{bidir}.
    \item The combination of checking and synthesizing means that there are multiple ways to create a typing judgment. For example, there are eight different rules for a judgment with two premises and one conclusion \citep{bidir}.   
   \item Dunfield and Krishnaswami \citep{bidir} defined general design criteria for a bidirectional type system. The design criteria are:
   \begin{itemize}
       \item Mode-correctness, no guessing of types.
       \item Completeness, all terms match at least one of the rules.
       \item Size, fewer rules are easier to work with.
       \item Annotation character, sensible annotations. Annotations should be lightweight, predictable, stable, and legible.
   \end{itemize}
   \item \citep{bidir} also presented a method for creating such a typing system, which is an alteration of the Pfenning recipe \citep{pfenning-recipe}. 
   The typing rules are divided into two forms of rules: introduction and elimination rules. Introduction rules introduce a connective in the conclusion, while an elimination rule eliminates a connective present in one of the premises. A connective connects multiple formulas, for instance, abstraction $\lambda x. e$ and let expressions $let x = e in e'$.
   \item Subsumption is an extra checking rule which accepts synthesizable expressions such as application $e_1 e_2$ \citep{bidir}. This is possible due to checking containing more information than synthesizing. Changing in the other direction is thus not possible. 
   
   
\end{itemize}
